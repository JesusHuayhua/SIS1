\documentclass[a4paper]{article}
% ----------------------------------------------------
%       PAQUETERIA
% ----------------------------------------------------
\usepackage[utf8]{inputenc}
\usepackage[T1]{fontenc}
\usepackage{makeidx}
\usepackage{graphicx}
\usepackage[spanish]{babel}
%Para modificar los margenes
\usepackage[left=2cm,right=2cm,bottom=3cm,top=3cm]{geometry}
%-------------------------------------------------------------------------------
\usepackage[table,xcdraw,dvipsnames]{xcolor}%Para poder agregar colores
%DEFINITION OF COLORS
\definecolor{PUCP}{HTML}{1034A6}
%-------------------------------------------------------------------------------
\usepackage[hidelinks]{hyperref}%usepackage of links
%-------------------------------------------------------------------------------
% Se uso en la portada de este documento para centra el título que tiene
\usepackage{setspace}
%-------------------------------------------------------------------------------
%				EXTRAS
\parindent = 0mm%Para eliminar la sangría de las páginas
%-------------------------------------------------------------------------------
% 					DEFINIR EL ESTILO DE LA PÁGINA
\usepackage[myheadings]{fancyhdr}% Configuraciones de stylo de página y cabecera
\pagestyle{fancy}%Para decir que vamos a trabajar con el style FANCY
\fancyhf{}%Para vaciar las configuraciones por defecto del style FANCY
\setlength{\headheight}{40pt}%Para agrandar la cabecera
% \lhead{\includegraphics[width=3cm]{ssd-vs-hdd.png}} %definimos la cabecera derecha
\rhead{\Large\thepage}
\chead{\LARGE\textsl{Tarea Específica}}
\renewcommand{\headrulewidth}{3pt}
\renewcommand{\headrule}{\hbox to\headwidth{\color{PUCP}\leaders\hrule height \headrulewidth\hfill}}
%-------------------------------------------------------------------------------

\title{Examen Final}
\author{Jesus Huayhua}

\begin{document}
  \begin{titlepage}
	\begin{center}
    % Nombre de la universidad
		{\Huge \textbf{Universidad Pontifícia Católica del Perú}}\\
		\vspace{3mm}
    %Facultad
		{\Huge \textbf{Facultad de Ciencia y Ingeniería }}\\
		\vspace{1cm}
    % Nombre del año
    {\Huge\textbf{Año de la unidad, la paz y el desarrollo}}\\
    \vspace{1cm}
		\begin{figure}[h]
			\centering
      %Imagen de la católica
			\includegraphics[width=15cm]{cover/logo_PUCP.png}
		\end{figure}
		\vspace{5mm}
    %Nombre del curso
		{\LARGE \textbf{\textsc{Sistema de Información 1}}}
		\vspace{2mm}
		\textcolor{PUCP}{\rule{\linewidth}{0.75mm}}\\
		\begin{spacing}{1}
      % Numero del informe
			\LARGE\textsc{Informe 1}
		\end{spacing}
		\textcolor{PUCP}{\rule{\linewidth}{0.75mm}}\\
		
		\vspace{1cm}
		\begin{flushleft}
      % Nombre del profesor
			{\Large\textbf{Profesor: } Cesar Aguilera}
			\vspace{2mm}
			
			{\Large\textbf{Integrantes:}}
      % Nombre de Los integrantes
			\begin{itemize}\Large
				\item[$ \bullet $] codigo\dotfill\textsl{Alumno}
				\item[$ \bullet $] codigo\dotfill\textsl{Alumno}
				\item[$ \bullet $] codigo\dotfill\textsl{Alumno}
				\item[$ \bullet $] codigo\dotfill\textsl{Alumno}
			\end{itemize}
		\end{flushleft}
		\vfill
		{\Huge \textbf{2023-1}}
	\end{center}
\end{titlepage}
	\newpage
  \tableofcontents
  \newpage
  \section{Introducción}
  \subsection{Tema}
  \subsection{Importancia}
  \subsection{Objetivo}
  \subsection{Estructura del informe}
  \section{CAPÍTULO 1: RECOLECCIÓN DE INFORMACIÓN}
  \subsection{Sección 1.1: Construcción del Cuestionario}
  \subsubsection{Explicación de las ventajas}
  \subsubsection{Explicación del método}
  \subsubsection{Cuadro resumen}
  \subsection{Seción 1.2: Entrevistas}
  \subsection{Seción 1.3: Documentación de la Entrevistas: Procesos}
  \section{ CAPÍTULO 2: ANÁLISIS DEL SISTEMA Y DISEÑO PRELIMINAR}
  

\end{document}